\documentclass[aspectratio=169]{beamer}

\usepackage{default}
\usepackage[utf8]{inputenc}
\usepackage[T1]{fontenc}
\usepackage[spanish]{babel}
\usepackage{graphicx}
\usepackage{wrapfig}
\usepackage{listings}
\usepackage{lipsum}

% ==================================================================================
% Tipos
% ==================================================================================
\usepackage[sfdefault]{roboto}  %% Option 'sfdefault' only if the base font of the document is to be sans serif
\usepackage[T1]{fontenc}



% ==================================================================================
%  Símbolos matemáticos
% ==================================================================================

\usepackage{amsmath}
\usepackage{amsfonts}
\usepackage{amsthm}

\def\N{\ensuremath{\mathbb{N}}}
\def\Z{\ensuremath{\mathbb{Z}}}
\def\Q{\ensuremath{\mathbb{Q}}}
\def\R{\ensuremath{\mathbb{R}}}
\def\C{\ensuremath{\mathbb{C}}}


% ==================================================================================
%  Colores
% ==================================================================================

\usepackage[]{xcolor}

\definecolor{azulUCA}{cmyk}{1.00, 0.00, 0.00, 0.51}
\definecolor{naranjaUCA}{cmyk}{0.00, 0.51, 1.00, 0.00}
\definecolor{grisUCA}{cmyk}{0.00, 0.00, 0.00, 0.65}
\definecolor{charcoal}{rgb}{0.21, 0.27, 0.31}
\definecolor{darkjose}{HTML}{4a4a4a}

% ==================================================================================
%  Algunos diseños
% ==================================================================================
\usepackage{epigraph}

\newcommand{\citainicial}[1]{
	\raya
	\begin{quotation}
		\color{charcoal}{\textit{#1}}
	\end{quotation}
	\raya
}

\def\raya{
	\par \hbox to
	\linewidth{\hss \vrule width \textwidth height 0.2pt depth 0.5pt}
	\par}

\makeatletter

\newcommand{\seccion}[1]{
	\section*{\bfseries\color{naranjaUCA}{#1}}
}

\newcommand{\subseccion}[1]{
	\subsection*{\bfseries\color{naranjaUCA}{#1}}
}

\setbeamercolor{frametitle}{fg=white,bg=azulUCA}

\usepackage[theorems,breakable]{tcolorbox}

\tcbuselibrary{theorems}

\newtcbtheorem{problema}{Problema}%
{
	colback=azulUCA!5,
	colframe=azulUCA,
	fonttitle=\bfseries
}{pr}

\newtcbtheorem{formulacion}{Formulación}%
{
	colback=naranjaUCA!5,
	colframe=naranjaUCA,
	fonttitle=\bfseries
}{for}

\newtcbtheorem{autores}{Autores}%
{
	colback=azulUCA!5,
	colframe=azulUCA,
	fonttitle=\bfseries
}{aut}

%\addtobeamertemplate{background canvas}{\transboxin}{}

% ==================================================================================
%  Definiciones
% ==================================================================================
\title{\color{azulUCA}\textbf{Breve introducción a la programación en paralelo}}
\subtitle{\color{azulUCA} José Carlos García}
\date{}

\setbeamertemplate{background}
{\includegraphics[width=\paperwidth,height=\paperheight]{background.jpg}} 

\usepackage{xcolor}
\usepackage{listings}
\setbeamerfont{frametitle}{series=\bfseries}
\definecolor{mGreen}{rgb}{0,0.6,0}
\definecolor{mGray}{rgb}{0.5,0.5,0.5}
\definecolor{mPurple}{rgb}{0.58,0,0.82}
\definecolor{backgroundColour}{rgb}{0.95,0.95,0.92}

\lstdefinestyle{CStyle}{
	backgroundcolor=\color{backgroundColour},   
	commentstyle=\color{mGreen},
	keywordstyle=\color{magenta},
	numberstyle=\tiny\color{mGray},
	stringstyle=\color{mPurple},
	basicstyle=\footnotesize,
	breakatwhitespace=false,         
	breaklines=true,                 
	captionpos=b,                    
	keepspaces=true,                 
	numbers=left,                    
	numbersep=5pt,                  
	showspaces=false,                
	showstringspaces=false,
	showtabs=false,                  
	tabsize=2,
	language=C
}



\begin{document}

\begin{frame}
	\titlepage
	\begin{center}
		\href{http://creativecommons.org/licenses/by-sa/3.0/es/}{\includegraphics[width=4em]{cc-by-sa}}
	\end{center}
\end{frame}

\begin{frame}
	\centering \LARGE \bfseries \color{naranjaUCA} Introducción
\end{frame}

% Ejemplos en la vida cotidiana.

\begin{frame}
	\frametitle{Ejemplos en la vida cotidiana}
	\begin{itemize}
		\item Supongamos que \textbf{tenemos un restaurante.}
		\pause
		\item Si tenemos \textbf{una persona trabajando a una velocidad media}, y nuestro restaurante \textbf{hay $n=10$ mesas ocupadas}, ¿Qué sucede?
		\pause
		\item Dado que tenemos una persona, y no un pulpo, \textbf{sólo puede atender a una persona a la vez.}
		\pause
		\item En este caso, tendríamos un ejemplo de \textbf{un trabajo que no se está \\realizando en paralelo (o de manera secuencial)}, y nuestro \\solitario camarero estará realizando $n$ operaciones él solito
		\pause
		\item \textbf{¿Cual es la mejor solución de nuestro problema?} \\¿Contratamos más camareros de velocidad media? \\¿Contratamos el mejor camarero del país?
	\end{itemize}
\end{frame}

\begin{frame}
		\centering \LARGE \bfseries \color{naranjaUCA} Aplicaciones en el mundo matemático
\end{frame}

% Operaciones con vectores
\begin{frame}[fragile]
	\frametitle{Operaciones con vectores}
	
	\begin{itemize}
		\item Sean $v, u \in \mathbb{R}^n$ si queremos calcular $w=u+v$ debemos calcular la suma de cada componente, esto es:
		\pause
		\item $w_i = v_i + u_i$ con $i = 1, 2, 3, ..., n$ (Observamos que cada $w_i$ es independiente del resto)
		\pause
		\item Por tanto, tenemos que hacer $n$ operaciones para obtener el valor de \\$w$. Sin embargo, si ejecutamos estas $n$ operaciones en paralelo, \\nos costaría casi lo mismo que hacer la suma de dos números.
		\pause
		\item Esta misma idea es aplicable también para calcular el producto \\de un número $\lambda$ y un vector $u$.
	\end{itemize}
\end{frame}


% Producto escalar
\begin{frame}[fragile]
	\frametitle{Producto escalar: No es paralelizable}
	
	\begin{itemize}
		\item En este ejemplo, veremos que \textbf{no siempre se puede paralelizar un problema}, y veremos uno de los problemas más comunes a la hora de paralelizar; \textbf{Las condiciones de carrera}
		\pause
		\item Sea $v, u \in \mathbb{R}^n$ \textbf{dos vectores a los que queremos calcular el producto escalar}, $s=u \cdot v$. Recordemos que el producto escalar se puede calcular como \\la suma del producto de las componentes.
		\pause
		\item Para programar esto, tenemos que ir sumando a una variable\\real $s$ el producto de todos los componentes, $v_i * u_i$.
		\pause
		\item Supongamos que queremos hacerlo en paralelo, y sucede \\que las componentes $i, j$ se ejecutan al mismo tiempo.
	\end{itemize}
\end{frame}

\begin{frame}[fragile]
	\frametitle{Producto escalar: No es paralelizable}
	
	\begin{itemize}
		\item Por tanto, al mismo tiempo se ejecutará $s = s + v_i * u_i$ y $s = s + v_j * u_j$.
		\pause
		\item Al realizarse al mismo tiempo, el valor $s$ que se suma al producto, será el mismo, por tanto, dado que sólo se asignará un valor de $s$ (Supongamos\\ que es $i$) el valor $v_j * u_j$ no se sumará a $s$.
		\pause
		\item Por tanto, el producto escalar no podemos paralelizarlo, pues \\\textbf{tenemos una condición de carrera.}
	\end{itemize}
\end{frame}

\begin{frame}[fragile]
	\frametitle{Otras aplicaciones}
	
	\begin{itemize}
		\item Resolución de sistemas de ecuaciones; Jacobi es fácilmente paralizable.
		\pause
		\item Criptografía; Encontrar primos, blockchain...
		\pause
		\item Cálculos con matrices.
		\pause
		\item Encontrar mínimos de funciones.
		\pause
		\item Encontrar gran cantidad de raíces de una función a la vez.
		\pause
		\item Acelerar método de diferencias finitas.
		\pause
		\item Trabajar con Big Data.\pause\\ Podemos procesar muchos datos a la vez.
		\pause
		\item Acelerar todas las tareas que se hacen por fuerza bruta;\\Romper contraseñas, encontrar combinaciones, ...
		\pause
		\item ...
	\end{itemize}
\end{frame}

\begin{frame}
	\centering \LARGE \bfseries \color{naranjaUCA} Algunos conceptos básicos
\end{frame}

\begin{frame}
	\frametitle{Conceptos básicos}

	\begin{itemize}
		\item Decimos que \textbf{un algoritmo de $n$ pasos se puede paralelizar} si cada $n$ iteración no depende del resto de iteraciones.
		\pause
		\item Formas de paralelización:
		\begin{itemize}
			\item \textbf{CPU:} Utilizando los hilos disponibles en el procesador de nuestro ordenador. Todos los hilos son de alto rendimiento, pero la cantidad de hilos es \\bastante pequeña.
			\pause
			\item \textbf{GPU:} Utilizando los hilos disponibles en la tarjeta gráfica de nuestro\\ ordenador. Los hilos tienen un menor desempeño que los de la \\CPU, sin embargo, contiene una gran cantidad de hilos.
			\pause 
			\item \textbf{Red:} Distribuimos el trabajo entre varios ordenadores a través\\ de una conexión de red.
		\end{itemize}
	\end{itemize}
\end{frame}

\begin{frame}
	\centering \LARGE \bfseries \color{azulUCA} Programar en paralelo
\end{frame}

\begin{frame}
	\centering \LARGE \bfseries \color{naranjaUCA} Trabajar con la CPU
\end{frame}

\begin{frame}
	\frametitle{Wolfram Language}
	
	\begin{itemize}
		\item Función \textit{ParallelX[]}.
		\pause
		\item \pause \textbf{Demo de \textit{ParallelX}}
		\pause
		\item Función \textit{Parallelize[]}
		\pause
		\item \pause \textbf{Demo de \textit{Parallelize}}
		\pause
	\end{itemize}
\end{frame}

% http://gforge.se/2015/02/how-to-go-parallel-in-r-basics-tips/
\begin{frame}
	\frametitle{R}
	
	\begin{itemize}
		\item \textit{lapply}: \pause Aplicar una lista a una función y obtener el resultado.
		\pause
		\item \textit{Librería paralell}: \pause Esta librería nos permite paralelizar funciones en R. \pause \\Tenemos dos funciones importantes; \textit{detectCores()} nos detecta el \\número de núcleos de nuestro ordenador. \pause\textit{makeCluster(corenum)} \\inicializa una instancia en paralelo, donde corenum es el \\número de núcleos que quieres usar.
		\item \textit{parLapply}: Igual que la función lapply, pero en paralelo.
		\item \textit{Otras funciones interesantes:} Paquete foreach, nos permite\\ejecutar una acción en una lista de elementos.
		\item \textbf{Demo}
	\end{itemize}
\end{frame}

\begin{frame}
	\frametitle{C: PThread \& OpenMP (Unix)}
	
	\pause
	\begin{itemize}
		\item La manera más convencional es usando pthread.h \pause
		
		\begin{itemize}
			\item Tenemos que incluir \textit{pthread.h} \pause
			\item Para ejecutar una función en paralelo, la llamamos con \textit{phtread\_create.}
			\item Si queremos esperar que un hilo termine antes de ejecutar algo, \\llamamos \textit{pthread\_join}
		\end{itemize}
	
		\item Con OpenMP todo esto se simplifica, \pause
		% http://lsi.ugr.es/jmantas/pdp/ayuda/omp_ayuda.php?ayuda=omp_directivas
		\begin{itemize}
			\item \#pragma omp parallel for\pause
			\item \#pragma omp parallel
			\item ...\pause
		\end{itemize}
	
		\item \textbf{DEMO}
	\end{itemize}
	
\end{frame}

\begin{frame}
	\centering \LARGE \bfseries \color{naranjaUCA} Trabajar con la GPU
\end{frame}

\begin{frame}
	\frametitle{Problemas}
	
	\begin{itemize}
		\item La GPU y la CPU por lo general, \textbf{NO COMPARTEN} memoria. \pause
		\item Copiar memoria de la CPU a la GPU (y viceversa) es \textit{lento}.\pause
		\item Por tanto, necesitamos unas funciones diferentes para reservar memoria para CPU y GPU. \pause
		\item Las funciones que definamos en la CPU, en general, no son\\ accesibles por la GPU. \pause
		\item Las variables que definamos en la CPU, en general, no son \\accesibles desde la GPU. \pause
		\item Existen compiladores libres, como OpenCL, sin embargo, el \\compilador propietario CUDA funciona mejor en gráficas\\ NVIDIA.
	\end{itemize}
\end{frame}

\begin{frame}
	\frametitle{CUDA}
	
	\begin{itemize}
		\item Es un lenguaje desarrollado por NVIDIA basado en C++, orientado a programar en GPU. \pause
		\item \textit{cudaMalloc: } Equivalente a malloc, pero para reservar memoria en la \\GPU. También tenemos \textit{cudaFree}.\pause
		\item \textit{cudaMemcpy: } Nos permite copiar memoria de la \\CPU a la GPU (y viceversa) \pause
		\item Para definir una función que se llame en la GPU, debemos\\ poner delante \textit{\_\_global\_\_} ; \pause \textit{\_\_global\_\_ void funcion(...)} \pause
		\item Para llamarla, escribimos \textit{funcion<<bloques, hilos>>(args)} \pause
		\item Wolfram Language, permite ejecutar código en CUDA.\pause
		\item \textbf{DEMO}
	\end{itemize}
\end{frame}

\end{document}
